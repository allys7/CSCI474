\documentclass[11pt,largemargins]{homework}
\usepackage{subcaption}
\usepackage{caption}
\usepackage{float}

% TODO: replace these with your information
\newcommand{\hwname}{Ally Smith}
\newcommand{\hwemail}{Section A}
\newcommand{\hwtype}{Lab}
\newcommand{\hwnum}{1}
\newcommand{\hwclass}{}
\newcommand{\hwlecture}{0}
\newcommand{\hwsection}{Z}

\newcommand{\code}{\texttt}

\begin{document}
\maketitle

\question*{Encryption using different ciphers and modes}
I tried the AES-128-CBC, DES-CBC, and DES3 encryption modes, each with a few
key values. For each of the encryptions I made, even a small difference in the
key made a significant change in the cipher text, demonstrating the avalanche
effect.

\question*{Encryption Mode --- ECB vs. CBC}
I encrypted the picture using they key and initial value from task 1. After
fixing the header to the bitmap file, I was able to see no patterns in the
encrypted image, seen below. The image that was generated looks to be
random `noise,' so no useful information can be derived from the picture.

\begin{center}
    \includegraphics[width=.5\textwidth]{pic_ciphered.jpg}

    Figure 1: Ciphered image
\end{center}

\clearpage
\question*{Encryption Mode --- Corrupted Cipher Text}
I suspect that using ECB on the corrupted cipher text will have the most
recoverable text, as there is no feedback, so the bit error will not propagate
beyond the block it is in. In CBC, the previous block's cipher text is XORed
with the decrypted block, so the bit error will affect the decrypted text in
the block \emph{after} that which the error occurred in the cipher text. In
CFB, I suspect that the error will greatly affect the deciphered text, as the
ciphertext block is used as feedback to the next block. Because feedback in OFB
is independent of the plaintext or ciphertext, it is possible that a small (1
bit) change in the ciphertext will not affect the decryption process at all.

After performing the encryptions, changing a single bit in the 30th byte, and
decrypting, I was surprised by some of the results. For all of the methods
used, over 90\% of the plaintext was recoverable. Below is a table showing how
many had changed in the decrypted plaintext from the plaintext (out of a
total of 446 characters).

\begin{center}
    \begin{tabular}{|c|c|}\hline
        ECB & 16 \\\hline
        CBC & 15 \\\hline
        CFB & 18 \\\hline
        OFB & 0 \\\hline
    \end{tabular}
\end{center}

\question*{Padding}

\question*{Programming using the Crypto Library}

\question*{Pseudo Random Number Generation}
\begin{alphaparts}
    \questionpart{Measure the Entropy of Kernel}

    \questionpart{Get Pseudo Random Numbers from \code{/dev/random}}

    \questionpart{Get Random Numbers from \code{/dev/urandom}}
\end{alphaparts}

\end{document}