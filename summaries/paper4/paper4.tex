\documentclass[12pt]{article}

\title{\bf A Summary of `A Longitudinal, End-to-End View of
the DNSSEC Ecosystem'}
\author{Ally Smith}
\date{due April 28, 2022}

\begin{document}
\maketitle{}

Much like some of the other papers we have read in this class, the sheer
magnitude of this study is impressive. However, something that sets this paper
apart from the rest is the extreme levels of misuse of the DNSSEC
infrastructure. With 31\% of domains failing to publish the relevant
information, to only 12\% of resolvers attempting to validate the records they
receive, this is far greater levels of noncompliance than in other papers.
Overall, the approach that they took in their research was very standard for a
project like this. They were able to survey large amounts of nodes, and checked
which of those they found were reporting that they supported DNSSEC, and which,
out of those, actually implemented it in a way that granted practical
improvements to security. I think this could possibly tie in with the first
research paper, in that it is possible that the maintainers of these keys are
simply ignorant to the proper ways to use DNSSEC, and the documentation around
it is not clear enough to users for the layman to set it up appropriately.
Another interesting finding from this paper was that certain domains share keys
for efficiency reasons. While this is an intentional decision made by the key
holders, the paper recommends against this as it `substantially increases
security risk.'
\end{document}