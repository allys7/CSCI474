\documentclass[12pt]{article}

\title{\bf A Summary of `An Empirical Study of Cryptographic Misuse in Android
Applications'}
\author{Ally Smith}
\date{due April 28, 2022}

\begin{document}
\maketitle{}

The Android mobile operating system provides common functionality to
cryptographic APIs, so that developers may more easily secure things like
passwords and personal information. However, this study finds that a substantial
number of developers still make mistakes that minimize the security overall.
The authors of the paper also identified factors that contribute to this level
of misuse, including missing lightweight security checks like in their program
CryptoLint, and the default behavior of the library not being recommended
practices. They believe that by including a set of tools to evaluate the
security of a developer's program, they are more likely to adhere to
cryptographic best practices. Additionally, by changing the default behavior of
the API to be more secure practices, it is more likely for a more layman
developer to follow good cryptographic security guidelines, as the program does
this by default now. They also highlight how much of the API is missing
documentation on how it is intended to be used, and they believe that adding
a `security discussion' would help address this. I believe that the
recommendations in this paper are all logical and would provide much-needed
security improvements to cryptographic APIs. Even during the short amount of
programming with cryptography in this class that we have done, I have seen first
hand how poorly documented some of these functionalities can be, and it would be
very beneficial to many developers if it was made easier to comply with these
standards.
\end{document}