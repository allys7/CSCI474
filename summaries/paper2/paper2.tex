\documentclass[12pt]{article}

\title{\bf A Summary of `Mining Your \\Ps and Qs: Detection of
Widespread Weak Keys in Network Devices'}
\author{Ally Smith}
\date{due April 26, 2022}

\begin{document}
\maketitle{}

This paper was quite interesting. I found it very impressive that they analyzed
the entire public IPv4 space and collected tens of millions of TLS certificates
and SSH keys. I was shocked at the number of private keys for the TLS and SSH
they were able to compute. It seems as if they conducted a comprehensive study
as to the vulnerabilities that enabled them to compute these keys. One of the
vulnerabilities that I found most interesting was the repeated keys due to low
system entropy. It makes sense that small embedded devices have a limited
entropy pool, due to the lack of user interaction. However, many of these
devices still require accessing \texttt{/dev/urandom} and therefore run into
issues. I am interested to see the presentation during class to see if they
cover any possible solutions to this problem. Overall, I think that the
methodology that they used to scan and analyze the keys is very promising for
future research. As mentioned in the paper, by scanning the keys that actually
exist in the world, researchers can find vulnerable keys far faster than
previously possible. In the past, individual hardware systems were reverse
engineered or flaws were observed and recorded by a user. I hope that the
presenters in class discuss some potential uses for this methodology.
\end{document}